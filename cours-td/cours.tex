\documentclass[12pt, a4paper]{article}
\usepackage[francais]{babel}
\usepackage{caption}
\usepackage{graphicx}
\usepackage{bigfoot}
\usepackage[T1]{fontenc}
\usepackage{listings}
\usepackage{geometry}
\usepackage{minted}
\usepackage{array,multirow,makecell}
\usepackage[colorlinks=true,linkcolor=black,anchorcolor=black,citecolor=black,filecolor=black,menucolor=black,runcolor=black,urlcolor=black]{hyperref}
\setcellgapes{1pt}
\makegapedcells
\usepackage{fancyhdr}
\pagestyle{fancy}
\lhead{}
\rhead{}
\chead{}
\rfoot{\thepage}
\lfoot{Martin Baumgaertner}
\cfoot{}
\renewcommand{\footrulewidth}{0.4pt}
\renewcommand{\headrulewidth}{0.4pt}
\renewcommand{\listingscaption}{Code}
\renewcommand{\listoflistingscaption}{Table des codes}
% \usepackage{mathpazo} --> Police à utiliser lors de rapports plus sérieux

\begin{document}
\begin{titlepage}
	\newcommand{\HRule}{\rule{\linewidth}{0.5mm}} 
	\center 
	\textsc{\LARGE iut de colmar}\\[6.5cm] 
	\textsc{\Large R316}\\[0.5cm] 
	\textsc{\large Année 2022-23}\\[0.5cm]
	\HRule\\[0.75cm]
	{\huge\bfseries Analyse de vulnérabilités}\\[0.4cm]
	\HRule\\[1.5cm]
	\textsc{\large martin baumgaertner}\\[6.5cm] 

	\vfill\vfill\vfill
	{\large\today} 
	\vfill
\end{titlepage}
\newpage
\tableofcontents
\listoflistings
\newpage
\section{TD 1 - 29 novembre 2022}
\subsection{Introduction aux commandes de bases}
\subsubsection{La commande HOST}
Cette commande permet de récupérer les informations DNS d'un domaine. 
Elle permet de récupérer l'adresse IP d'un domaine, ainsi que les serveurs
DNS associés. La différence entre \textbf{uha.fr} et \textbf{www.uha.fr} est
que le premier est le domaine, alors que le second est un sous-domaine.

\subsubsection{La commande DIG}
La commande \textbf{dig} permet aussi d'intérroger un DNS. Elle donne des informations similaires
à la commande \textbf{host} mais sous un autre format. En faisant \textbf{dig -h}, on peut voir
les options disponibles. On peut par exemple faire \textbf{dig +trace} pour voir
l'historique des requêtes DNS. On peut aussi faire \textbf{dig +short} pour n'afficher
que les résultats. 

\subsubsection{La commande NSLOOKUP}
La commande \textbf{nslookup} permet de faire des requêtes DNS. On peut par exemple
faire \textbf{nslookup uha.fr} pour récupérer les informations DNS du domaine \textbf{uha.fr}.
On peut aussi faire \textbf{nslookup -type=mx uha.fr} pour récupérer les serveurs
mail du domaine \textbf{uha.fr}.

\subsubsection{La commande WHOIS}
La commande \textbf{whois} permet de récupérer les informations d'enregistrement d'un domaine.
On peut par exemple faire \textbf{whois uha.fr} pour récupérer les informations d'enregistrement
du domaine \textbf{uha.fr}. On peut aussi faire \textbf{whois -h whois.ripe.net uha.fr} pour
récupérer les informations d'enregistrement du domaine \textbf{uha.fr} sur le serveur
\textbf{whois.ripe.net}.

\subsection{Le scan de réseaux}
\subsubsection{La commande NMAP}
La commande \textbf{nmap} permet de scanner un réseau. Si on teste la commande :
\textbf{nmap -sUV -F 192.168.2.23}, on peut voir que la commande \textbf{nmap} permet de scanner
un réseau en utilisant les protocoles UDP, TCP et ICMP.





\end{document}